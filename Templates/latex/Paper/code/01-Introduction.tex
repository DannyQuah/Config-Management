% @(#) 01-Introduction.tex
%
% Last edited: Thu Jul 18 07:00 2018 - Danny Quah (dq@PBOOK)
% $
% Revision History:
%  % Thu Jul 18 07:00 2018 - Danny Quah (dq@pbook)
%    First draft
% $
% $Log$
%

{World order}---sustained joint economic prosperity,
together with peace and stability across the community of
nations: That is how many people view
global society ought to be and, indeed, mostly is.

After all, the world has been on a good run in modern times, despite
some well-known disruptions. Steve Pinker.

For many other observers, however, such a state of world order is an anomaly.
Instead, these observers tend to think the natural state of relations
across nations
as a situation fraught with suspicion, fear, and conflict.

There are different reasons why someone might think this.

\section{``The world is zero-sum''}

A first, obvious explanation stems from some of the rhetoric of
national leaders.
On 20 January 2017 Donald Trump became the 45th President of the US.
From the steps of the US Capitol, Trump
used his Inaugural Address to describe how Americans
``have made other countries rich while the wealth, strength,
and confidence of our country has dissipated over the horizon'',
To defend the nation, Trump asked Americans to
``protect our borders from the ravages of other
countries making our products, stealing our companies,
and destroying our jobs''.
Trump portrayed nations elsewhere having grown richer
from Americans' becoming poorer:
``the wealth of our middle class has been
ripped from their homes and then redistributed all across the world''.
It follows as a matter of logic from this depiction
that the entire community of nations cannot jointly prosper.
The world is zero-sum, so one part of the world
can gain only when another loses.
No matter how reasoned and tolerant any nation might be,
the objective reality is that by allowing others to prosper,
it itself suffers.
Cooperation can at best only ever be a knife-edge proposition as
strict win-win outcomes are never available.

But however uniquely evocative and memorable Trump's
inaugural address,
the zero-sum thinking it espouses is commonplace
\citep{Rachman-G-2011-Zero-Sum-Future}.
Indeed, the ``if you gain, I lose'' perspective attracts great
intuitive and intellectual appeal both.
This then is a second reason---deeper and more sustained---for
why many observers consider unlikely world order anomalous.
It is that the appeal of zero-sum thinking is visceral and
surfaces implicitly in many unlikely situations.

League tables---who is number 1 and who
therefore gets bragging rights;
which Premier League football team has risen and which
fallen this season; what university Economics department
came ahead of the next---provide vivid
examples of the instinctual appeal in a zero-sum view
of engagement and competition.
In all these cases, for any participant to rise (win),
someone else must fall (lose); there are no win-win outcomes.

Many observers not only fail to question the usefulness of such an
arrangement but instead welcome such a setup for its simplicity and
transparency.
The league-table structure allows easy mental organisation of
what would otherwise be a complicated collection of different players
with multiple, not-directly-comparable attributes.

If the world were, in reality, zero-sum, then no amount of reason
and tolerance could guide the collection of players to a
collaborative outcome.
Even if agents, players, or nations seek wherever they can
to benefit others, a zero-sum world means they can only do so
by harming themselves.
In contrast, however, shoehorning the domain of engagement
into a zero-sum structure---a league table---even when structure
provides mutual benefit instead implies a
rejection of the use of reason and tolerance even when these might help.
Given a league table, it is immediate
where any observer's loyalty lies.


In either of the two cases considered, however,
the logic leads to unlikely world order.
The world might be, in its underlying structures,
truly zero-sum---as Trump and others say.
Or societies might have simply
and unnecessarily shoehorned their
understanding of engagement with one another into a zero-sum game---as
in the example of league tables.
Either way, when that thinking is applied to the problem
of world order,
self-seeking nations---concerned rightly for the well-being of their
peoples---of course find cooperation and collaboration difficult.

But, while extreme in its starkness,
strict zero-sum reasoning has no monopoly over the
conclusion that world order is anomalous.


\section{Major Analytical Frameworks and Insight}

Consider two important and prominent examples
from the field of international Relations.
First, encapsulated in a set of ideas known broadly as \textsl{Realism},
concludes that world order
is an unstable state of affairs, and that
small perturbations in the landscape of global configuration
will precipitate a slide away from any initial equilibrium
showing peace and cohesion.

Nations seek to maximise the chances of their self-survival.
The way to do that is to amass maximum power of all kinds---military,
economic, scientific, technological, control of resources, among them.
Since, in this situation, as a matter of logic
the concept of \textsl{maximum} can only ever be a relative one,
this thinking also implies that undermining others' power is as good as
raising one's own, and thus is part of the pursuit of
optimising one's position in the landscape of power.

Next, observe that for all practical purpose
no overarching entity oversees
the international community of sovereign states.
There is no all-powerful body that has both
legitimacy and authority to command
the international community or to control any of its constituents.
In this club that contains everyone, there are no club rules.

Should the member states of this community choose to misbehave,
then along a wide range of possibilities
no law exists to regulate and restrain that behaviour.
Indeed, even should one state or group of states take upon itself
the responsibility of seeking to curb such misbehaviour,
there is no presumption that that action will attract approbation
from the international community.
Indeed, seeking to controls what others do,
without a clear framework of
what is acceptable and what isn't, simply runs the risk of
being seen as
encroaching on others' sovereignty.
Whether an action is harmful to others
or, conversely, is an attempt to bring wrong-doers in line,
all such actions
on the international stage might only ever be viewed as interference,
and therefore ultimately only self-serving.
Every action is suspect.

In this reasoning, corrosive and adversarial behaviour
among nations is not only possible,
it is inevitable.
When a nation scans the global landscape, it will realize
its well-being can improve
only by its asserting control and ownership over ever
greater resources---even
as doing so renders those
same resources no longer available to all others.
This is because those resources are, in economic language, rival:
Their use and consumption by one nation removes the possibility of other
nations doing the same.
Competition for resources, therefore, makes conflict unavoidable.

This Realist view of relations between nations is compelling
in its internal logic. It builds on the zero-sum thinking
earlier described but of course further provides a rich tapestry
of analysis for thinking about world order.

To turn to the second important strand of thinking in the
International Relations domain,
begin by noting that all the same conflicts arise \textsl{within}
a nation
as previously figured in the
Realists' description of engagement between nations.
The key difference, however, between the foreign and the domestic
application of these ideas rests on how
in many cases a nation state will have established internal
institutions---laws, norms, and conventions---to
manage this problem of otherwise-inevitable conflict.
Reason and tolerance can be harnessed to build institutions and
civil society to foster collaboration and thus order, instead of
permitting conflict to remain a constant in engagement.
That these institutions might be raised to an international level
and succeed internationally as they do domestically is
\textsl{Liberalism}.

In the domestic setting these institutions work
through a range of different channels.
The society-wide benefits to them might be obvious,
perhaps because they prevent deadweight loss.
In that case, the institutions
are themselves a social gain, so that even the most
determinedly self-seeking of citizens will find
cooperation beneficial.
But, more, even when the gain to all might not be plain,
domestic institutions can still work to
enforce socially efficient outcomes.

First, these institutions can be set up to
wield power sufficient to punish deviation
and reward compliance.
Rule of law and systems of punishment are instances
of such institutional arrangement.
A self-seeking agent, carrying out the calculations
on costs and benefits, will then adopt collaborative behaviour.
In other words, they will obey the law.

Second, domestic institutions might attract legitimacy in the eyes
of their constituents if the latter feel they have been an
integral part of a process that put into position
those institutions in the first place.
Individual agents then, even when doggedly self-serving,
can decide to hew to collaborative signals from their domestic institutions.
In many societies, democracy is viewed as part of such a
legitimizing process. In other societies, legitimacy might come
about in yet other ways, perhaps by those institutions
drawing on a history of credible delivery to their stakeholders.

Third, domestic institutions can leverage dynamic commitment.
There is a difference between short-term inspection
and long-term calculation:
Self-interested citizens looking forwards over a
time horizon relevant
to them---perhaps a time duration over the rest of their lifetime---can
well end up doing a cost-benefit calculation that shows
net gain to voluntary collaboration.
This might be the case
even though to an outside observer, it might seem that individual
collaboration, and thus social order, has come from either
coercion by domestic institutions or
the individual irrationally neglecting benefits to conflict and competition.

The difficulty for world order, in this
institutions-centered perspective,
is that the cross-nation context,
without further institution-building, provides none of the conditions
that allow domestic institutions to succeed---rule of law;
legitimacy through process; or dynamic commitment.

For Liberalism the challenge is to show how such
international institutions can, indeed, be provided, so that
the international community can be as well-ordered as successful
domestic ones.
Paradoxically, one way to achieve that is, not through
ever greater equality between nations, but instead through one
nation wielding overwhelming power,
except in a benevolent hegemonic way rather than as an imperialistic
power
\citep{Ikenberry-GJ-2015-The-Future-of-Liberal-World-Order,%
Kindleberger-C-1973-World-in-Depression%
}.
That hegemon is the entity that builds and enforces
a rules-based liberal order,
that in turn then allows the benefits of collaboration to follow
and
a world order, with peace and stability, to emerge.
Without such a hegemon, however, again the prospects for world order
remain dim.


\section{The Narrative of Power}

This discussion of how world order is an anomaly has come inextricably with
a narrative of power.

However power needs to be defined for different specific analyses,
it is power with which nations play their zero-sum game,
if indeed the world is zero-sum.
It is power that provides a convenient metric on which
observers can construct their league table for nations.
For Realists, it is power that nations are driven to acquire so
those nations can wield it to raise their well-being and their likelihood
for survival.
Indeed, even for those who adhere to Liberalism, it is power
that a benevolent hegemon needs to use to convince the international community
to adhere to a liberal rules-based order.

In other words, in conventional analysis, across a range of perspectives,
power determines world order.

It is at this point in the logic that this book
proposes a departure from established thinking.
In what follows, I seek to develop an economic approach to world order.
In this perspective, what international system emerges
will indeed hinge still on the international landscape of power.
However, I will argue that that configuration of power can
at most be only part of
the determining environment for world order.

The model I will present has as its underlying basis a view
of world order as a commodity to be determined in a metaphorical
marketplace (Fig.~\ref{Fig:WOMarket}).
As in any market, there will be supply and demand.
In the model the configuration of power across nations is a statement about
only the supply side of the market for world order.
And, again as in any market, supply and demand jointly determine
outcomes.

\begin{figure}
\begin{minipage}[c]{0.9\textwidth}
\begin{tikzpicture}[scale=1.0]
\input{\DirTikz/world-order-demand-supply}
\end{tikzpicture}
\end{minipage}%
\caption{
World Order: Demand and Supply, with equilibrium
World Order as indicated at the intersection
}
\label{Fig:WOMarket}
\end{figure}

\section{Demand and Supply}

What constitutes the demand side for world order?

The easy answer is that it is the entire community of nations.
It is this collection of players
that endures or enjoys the consequences
of whatever world order is put in place.
But of course included in this group of nations are also those
who might be the suppliers of order.

Indeed, in ordinary markets, as technologies and market preferences evolve,
it is not unusual to see agents on the demand side---who happen to be
on the appropriate margin---switch to become suppliers, or vice versa.
As the price for illegal cocaine in US markets skyrocketed in the 1980s,
Central American farmers, who had previously only used coca themselves
in the harsh agricultural climate, became drug cartel kingpins, supplying
cocaine to high-paying consumers in New York City.
So too, as the global economy and the international system evolve, observers
should not be surprised to see nations such as China---previously a consumer
of the rules of the international trading system---switch to become instead
standard-setters and rule-enforcers, i.e., supply-side providers.
Such a shift is nothing more than the natural consequence of
evolving demand and supply.

The recognition of this possibility also makes a general point. Just as in
general equilibrium theory in modern economics, analysis of the
market for world order needs to be sensitive to how the demand
and supply sides are only functional descriptors, not permanent identifiers.

Thinking of demand and supply for world order in functional terms also
helps resolve a natural challenge to this market framework I propose:
How could the demand
side---the hundreds of lesser nations in the world---be part of the
determination of world order?
Surely it must be only the Great Powers---the US certainly; once perhaps the UK
or France or Germany or the Soviet Union; now more likely China---that
could possibly have the capacity, technology, and power to decide on
peace, stability, and collaboration possibilities across all nations?

To understand the issues, a useful analogy is the market for taxis.
Conventions, rules, and regulations differ across nations, of course, but
think only of demand and supply for taxi rides. On the one hand, it
is a truism that only taxi companies can provide rides---they
have the heavy metal, the cars,
the drivers, and the petrol conveniently already in gas-tanks.
However, if the market were just taxi companies---the supply side---and
there were no consumers looking to get from one point to another, no
taxi rides would, in fact, take place.
More that that, without the mass of consumers, today's
car-rides industry would not have surge pricing,
ride-sharing, or on-demand transport as innovations off of
taxi companies supplying only the heavy iron of taxi rides.
And this happened without taxi-ride consumers having
to organise into a union of users, form coalitions to block roads,
sign petitions, and so on, but instead merely through articulating
anonymous market demand.

So too---in the market interpretation I develop in the chapters that
follow---will world order have to shift to meet the legitimate demands
of the world's consumers of the international system
(Fig.~\ref{Fig:WODemandSide}).

\begin{figure}\label{fig:World-Tightest-Cluster}
\caption{%
The world's 2015 tightest population cluster:
That circle centred in Shan State, eastern Myanmar, with a radius
3,300km on the surface of the 3-dimensional planet.
}
\includegraphics[width=1.1\linewidth]%
 {\DirImages/2015.09.22-Danny.Quah-3d-population-cluster-side-More-People-Inside-Democracy}
 \label{Fig:WODemandSide}
\end{figure}

The market interpretation of world order also gives insight
into optimal strategies for lesser powers or small states among the
community of nations.
A power narrative for world order suggests that the Great Powers
always get their way, or as Thucydides put it,
``right, as the world goes, is only in question between equals in power,
while the strong do what they can and the weak suffer what they must''.
In contrast, in the demand and supply picture of Fig.~\ref{Fig:WOMarket},
any smaller state on the demand side can choose between~(a) shopping around,
i.e., searching across alternative providers of world order for the
best deal; (b) leverage the slope of the supply curve, i.e., draw on
the elasticity of supply or the responsiveness of a provider to
obtain alternative options.

Finally, the market interpretation of world order can be used
to analyse historical turning points in the international system,
to provide better insight on global power shifts more generally.

\section{New forms of Leadership}

Leadership is just being the one who calls the meeting. [$\dots$]


\section{Future World Order. By Task}

``Why would you need an aircraft carrier if all you want to do
is regulate financial markets?''

A future world order as a gig economy. [$\dots$]

%
% ^L
% Local Variables:
% mode: TeX
% end:
% eof 01-Introduction.tex
