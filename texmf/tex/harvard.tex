\def\BibTeX{{\rm B\kern-.05em{\sc i\kern-.025em b}\kern-.08em
    T\kern-.1667em\lower.7ex\hbox{E}\kern-.125emX}}

\documentstyle[harvard]{article}
\title{The {\em Harvard} Family of Bibliography Styles}
\author{Peter Williams \\ (peterw@archsci.arch.su.oz.au)}
\begin{document}
\bibliographystyle{agsm}
%\citationstyle{agsm}
\maketitle
\section{Introduction}
This document describes the {\em harvard} family of bibliographic styles which
are provided in addition to those described in \citeasnoun{latex} and \citeasnoun{btxdoc}.
This style is primarily intended for use with the \BibTeX\ bibliographic
database management system.
However, provision is also made for hand coding of bibliographies.
\section{Citations}
There are two primary forms of citation in the {\em harvard} style dependent
upon whether the reference is used as a noun or parenthetically.
Additionally, where there are more than two authors, all authors are listed in
the first citation and in subsequent citations just the first author's name
followed by `et al.' is used.
The following example from \citeasnoun{agsm}\ illustrates these points.
\begin{quote}
The major improvement concerns the structure of the interview
(Ulrich~\& Trumbo~1965, p.~112) \ldots .
Later reports (Carlson, Thayer, Mayfield~\& Peterson 1971) record greatly 
increased interviewer reliability for structured interviews.
Wright (1969, p.~408) comments that `undoubtedly interviewer skill is
directly related to the validity, quantity and quality of the interview output',
and this would suggest some sort of interviewer training is called for.
Rowe (1960), for example, found that trained interviewers are better able to
evaluate applicants with some measure of reliability.
In addition Wexley, Sanders~\& Yukl (1973) showed that by extensive interviewer
training all significant contrast effects could be eliminated.
The results of the 1971 study (Carlson et al. 1971) are still relevant, but
efforts to~\ldots.
\end{quote}

To facilitate using a citation as a noun a new command
{\bf $\backslash$citeasnoun} has been created which has the same syntax as the
{\bf $\backslash$cite} command except that multiple citations are {\em not}
permitted.
The effect of this command is that
\begin{verbatim}
As \citeasnoun{btxdoc} and \citeasnoun[Annex~B]{latex} describe \ldots
\end{verbatim}
produces
\begin{quote}
As \citeasnoun{btxdoc} and \citeasnoun[Annex~B]{latex} describe \ldots
\end{quote}
whereas
\begin{verbatim}
The \BibTeX\ \cite{btxdoc} and \LaTeX\ \cite[Annex~B]{latex} manuals \ldots
\end{verbatim}
produces
\begin{quote}
The \BibTeX\ \cite{btxdoc} and \LaTeX\ \cite[Annex~B]{latex} manuals \ldots
\end{quote}

Where appropriate, citations are abbreviated automatically after the first
reference when bibliographies are produced by \BibTeX.
Provision is also made for this feature to be accessed during manual coding.

In addition to these primary forms of citation, the citation commands
{\bf $\backslash$citeyear} and {\bf $\backslash$citename} are provided.
{\bf $\backslash$citeyear} behaves like the
{\bf $\backslash$cite} command except that only the year portion of the
citation label is used.
For example,
\begin{verbatim}
\citeyear{btxdoc,latex}
\end{verbatim}
produces \citeyear{btxdoc,latex}.
{\bf $\backslash$citename} behaves like the
{\bf $\backslash$citeasnoun} command except that only the author name(s)
(unabbreviated) portion of the citation label is used.
For example,
\begin{verbatim}
\citename{btxdoc}
\end{verbatim}
produces
\begin{quote}
\citename{btxdoc}.
\end{quote}
The use of this command does not trigger the use of abbreviated citations for
subsequent {\bf $\backslash$citeasnoun} and {\bf $\backslash$cite}
references.

\section{Styles}
\subsection{Bibliography Styles}
There are four bibliography styles currently available within the
{\em harvard} family, {\bf agsm} (used in this document) which is based on 
\citeasnoun[pp.~95--98]{agsm}, {\bf dcu}
which is based upon the conventions in use in the Design Computing Unit,
Department of Architectural and Design Science, University of Sydney,
{\bf kluwer} which aspires to conform to the requirements of Kluwer Academic
Publishers and {\bf nederlands} which conforms to Dutch conventions.
They are invoked by the {\bf $\backslash$bibliographystyle} as described in
\citeasnoun[p.~74]{latex} and effect the layout of the entries in the bibliography.

\subsection{Citation Styles}
There are two citation styles currently available within the {\em harvard} 
family, {\bf agsm} (used in this document) and {\bf dcu} which for the previous
example would produce:
\begin{quote}\citationstyle{dcu}
The \BibTeX\ \cite{btxdoc} and \LaTeX\ \cite[Annex~B]{latex} manuals \ldots
\end{quote}
and for multiple citations such as
\begin{verbatim}
  The original documentation \cite{btxdoc,latex} say \ldots
\end{verbatim}
the {\bf agsm} citation style produces
\begin{quote}\citationstyle{agsm}
The original documentation \cite{btxdoc,latex} say \ldots
\end{quote}
and the {\bf dcu} citation style produces
\begin{quote}\citationstyle{dcu}
The original documentation \cite{btxdoc,latex} say \ldots
\end{quote}
The default citation style is {\bf agsm} and both styles have no effect on the
appearance of the {\bf $\backslash$citeasnoun} citation format.

These styles are invoked by the {\bf $\backslash$citationstyle} command,
for example:
\begin{verbatim}
  \citationstyle{agsm}.
\end{verbatim}
Because these styles affect the format of parenthetical citations, this command 
should appear before any {\bf $\backslash$cite} commands.

\section{Doing It By Hand}
Hand coding is accomplished much the same as described in \citeasnoun[p.~73]{latex}
except that the new command {\bf $\backslash$harvarditem} is used in place
of {\bf $\backslash$bibitem}.
The syntax of this command is
\begin{quote}
{\bf $\backslash$harvarditem} [{\em abbr-citation}]\{{\em full-citation}\}\{{\em citation-year}\}\{{\em cite-key}\}
\end{quote}
where
\begin{description}
\item[{\em abbr-citation}] is the (optional) abbreviated citation
(minus the year) to be used in the text
subsequent to the first mention of a particular reference,
\item[{\em full-citation}] is the full citation (minus the year)
to be used in the text
on the first mention of a particular reference,
\item[{\em citation-year}] the year portion of the citation including any
suffices required to disambiguate citations, and
\item[{\em cite-key}] is the key used in the {\bf $\backslash$cite} and
{\bf $\backslash$citeasnoun} commands.
\end{description}

\section{Acknowledgement}
The motivation for this style came from Fay Sudweeks of the Design Computing
Unit who also originated the formats for the {\bf dcu} style and proofread
their implementation.

The {\bf nederlands} bibliographic style was implemented by Werenfried Spit
(spit@vm.ci.uv.es).

The idea for {\bf $\backslash$citeyear} came from Renate Schmidt
(Renate.Schmidt@mpi-sb.mpg.de).

\bibliography{harvard}
\end{document}
